% rsheet.tex - R reference sheet
%
% Permission is granted to copy, distribute and/or modify this document under
% the terms of the GNU Free Documentation License, Version 1.3 or any later
% version published by the Free Software Foundation; with no Invariant
% Sections, no Front-Cover Texts, and no Back-Cover Texts.  A copy of the
% license is included in the file named ``LICENSE''.
%
% Acknowledgments:
%
% The page layout is based on Winston Chang's LaTeX cheat sheet.  Thanks to
% the folks at tex.SE for their help with the TeX code.
%
\documentclass[a4paper,landscape]{article}
\usepackage{multicol}
\usepackage[landscape]{geometry}
\geometry{top=1cm,left=1cm,right=1cm,bottom=1cm}

% \documentclass[a4paper]{article}
\usepackage{mathtools}
\usepackage{tikz}

% \drawarray{rows}{cols}{xdim}{xdim-shift}
\newcommand{\drawarray}[4]{%
  \draw (0,0) grid (#2,#1);
  \ifnum #3 > 0
  \foreach \x in {1,...,#3}
  \draw[xshift=#4 *\x cm, yshift=#4 *\x cm] (0,#1) -- (#2,#1) -- (#2,0);
  \fi}

% \drawtable{rows}{cols}{xdim}{xdim-shift}
\newcommand{\drawtable}[4]{%
  \drawarray{#1}{#2-1}{#3}{#4}
  \draw (-1,0) grid (0,#1-1);
  \draw (-1,#1-1) -- (0,#1);}

% \drawfactor{rows}
\newcommand{\drawfactor}[1]{%
  \drawarray{#1}{1}{0}{0}
  \draw (0,0) -- (1,1);}

% \drawdataframe{rows}{cols}
\newcommand{\drawdataframe}[2]{%
  \foreach \x [count=\xi] in {#2}{
    \if\x f
    \draw (\xi-1,0) -- (\xi,1);
    \draw (\xi-1,1) -- (\xi,1);
    \fi
  };
  \draw (0,0) grid [ystep=#1-1] (\xi,#1-1);
  \draw (0,#1-1) rectangle (\xi,#1);}

\newcommand{\drawaxes}{%
  \draw[->] (0,0) -- (0,2);
  \draw[->] (0,0) -- (2,0);
  \draw[->] (0,0) -- (1.5,1.5);}

\def\iconscale{.23}             % scale factor
\def\extradimsep{.2}            % separation for extra dimensions in cm
\def\mathbaselineshift{-.6ex}   % adjust baseline in math mode

\newcommand{\deficon}[3]{%
  \DeclareRobustCommand{#1}[#2]{%
  \ifmmode                      % needs \DeclareRobustCommand
  \def\iconbaseline{%
    \pgfpointadd{\pgfpointanchor{current bounding box}{center}}%
    {\pgfpoint{0}{\mathbaselineshift}}}%
  \,
  \else
  \def\iconbaseline{\pgfpointanchor{current bounding box}{south}}%
  \fi
  \begin{tikzpicture}[semithick]
    \pgftransformscale{\iconscale}
    #3
    \pgfsetbaselinepointlater{\iconbaseline}%
  \end{tikzpicture}%
  \ifmmode\,\fi
  }}

\deficon{\Array}{3}{\drawarray{#1}{#2}{#3}{\extradimsep}}
\deficon{\Table}{3}{\drawtable{#1}{#2}{#3}{\extradimsep}}
\deficon{\Factor}{1}{\drawfactor{#1}}
\deficon{\Dataframe}{2}{\drawdataframe{#1}{#2}}
\deficon{\Axes}{0}{\drawaxes{}}

\newcommand{\List}[1]{\Big[#1\Big]}

% use these when using icons in in-line math to get a better baseline
\newcommand{\nodepth}[1]{{\raisebox{\depth}{#1}}}
\newcommand{\descenderdepth}[1]{%
  \raisebox{\dimexpr\depth-\fontchardp\font`y}{#1}}

% \begin{document}

% \end{document}


% Redefine section commands to use less space
\makeatletter
\renewcommand{\section}{\@startsection{section}{1}{0mm}%
                                {-1ex plus -.5ex minus -.2ex}%
                                {0.5ex plus .2ex}%x
                                {\normalfont\large\bfseries}}
\renewcommand{\subsection}{\@startsection{subsection}{2}{0mm}%
                                {-1explus -.5ex minus -.2ex}%
                                {0.5ex plus .2ex}%
                                {\normalfont\normalsize\bfseries}}
\renewcommand{\subsubsection}{\@startsection{subsubsection}{3}{0mm}%
                                {-1ex plus -.5ex minus -.2ex}%
                                {1ex plus .2ex}%
                                {\normalfont\small\bfseries}}
\makeatother

% Don't print section numbers
\setcounter{secnumdepth}{0}

\pagestyle{empty}

\setlength{\parindent}{0pt}
\setlength{\parskip}{0pt plus 0.5ex}

\newcommand{\Maps}{\;\;\rightarrow \;\;}

\begin{document}

\raggedright
\small

\begin{multicols*}{3}
\setlength{\premulticols}{1pt}
\setlength{\postmulticols}{1pt}
\setlength{\multicolsep}{1pt}
\setlength{\columnsep}{2pt}

\begin{center}
  \Large{\textbf{
      A Visual Guide \\ to Split-Apply-Combine in R}} \\
\end{center}

\section{Legend}
The symbols used in this document are as follows:\\
\begin{tabular}{ll}
  \Factor{3}
  & Factor \\
  \Array{3}{1}{0} \Array{3}{2}{0} \Array{2}{2}{2}
  & Vector or array \\
  \Table{3}{3}{0} \Table{2}{2}{2}
  & Table \\
  \Dataframe{3}{v,v} \Dataframe{3}{v,v,f} \Dataframe{2}{v,v}
  & Data-frame \\
  \nodepth{$\!\List{\Dataframe{3}{v,v,v},\Factor{3},\Array{2}{1}{0},\dotsc}$}
  & List
\end{tabular}

\section{Reshaping}

\begin{description}

\item[as.matrix] Coerces a data-frame to a matrix.
\[
\Dataframe{3}{v,v,v} \Maps \Array{3}{3}{0}
\]

\item[as.table] Coerces an array to a contingency table.
\[
\Array{3}{3}{0} \Maps \Table{3}{3}{0}
\]

\item[as.data.frame] Coerces an array or contingency table to a
  data-frame.
\begin{align*}
  \Array{3}{3}{0} &\Maps \Dataframe{3}{v,v,v} \\
  \Array{3}{3}{2} &\Maps \Dataframe{3}{v,v,v,v,v,v} \\
  \Table{3}{3}{2} &\Maps \Dataframe{3}{f,f,f,v}
\end{align*}

\item[c] Coerces an array or contingency table to a vector.
\[
  \Array{3}{3}{2}\ \text{or}\ \Table{3}{3}{2} \Maps \Array{3}{1}{0}
\]

\item[unlist] Coerces a data-frame to a vector.
  \[ \Dataframe{3}{v,v,v} \Maps \Array{3}{1}{0} \]

\item[stack] Concatenates the columns of a data-frame into a single column
  along with a factor.  Non-vector columns are dropped.
  \[ \Dataframe{3}{v,v,v} \Maps \Dataframe{3}{v,f} \]

\item[unstack] Reverses the effects of \verb!stack!.
  \[ \Dataframe{3}{v,f} \Maps \Dataframe{3}{v,v,v} \]

\end{description}

\section{Splitting}

\begin{description}

\item[as.list] Coerces a data-frame to a list of vectors.
  \begin{align*}
    \Array{3}{3}{2} &\Maps \List{\Array{1}{1}{0},\Array{1}{1}{0},\dots} \\
    \Dataframe{3}{v,f,v}
    &\Maps \List{\Array{3}{1}{0},\Factor{3},\Array{3}{1}{0}}
  \end{align*}

\item[split] Divides a vector or data-frame into the groups defined by a
  factor.
  \begin{align*}
    \Array{3}{1}{0} + \Factor{3}
    &\Maps \List{\Array{2}{1}{0},\Array{3}{1}{0},\dots} \\
    \Dataframe{3}{v,v,v} + \Factor{3}
    &\Maps \List{\Dataframe{2}{v,v,v},\Dataframe{3}{v,v,v},\dots}
  \end{align*}

\item[unstack] Splits a vector or the column of a data-frame into the groups
  defined by a factor.
  \begin{align*}
    \Dataframe{3}{v,f}\ \text{or}\ \List{\Array{3}{1}{0},\Factor{3}}
    &\Maps \List{\Array{2}{1}{0},\Array{3}{1}{0},\dots}
  \end{align*}

\end{description}

\section{Applying}

Some of the following functions may return an array instead of a list
depending on the output of $f$.

\begin{description}

\item[tapply] Applies a function to each cell of a ragged array.
  \[
  \Array{3}{1}{0} + \List{\Factor{3},\Factor{3},\dots} + f
  \Maps \List{f\Big(\Array{3}{1}{0}\Big),f\big(\Array{2}{1}{0}\big),\dots}
  \]

\item[apply] Applies a function to margins of an array.
  \begin{align*}
    \Array{2}{2}{2} + \Axes + f 
    &\Maps
      \List{f\big(\Array{2}{1}{2}\big),f\big(\Array{2}{1}{2}\big),\dots}
      \ \text{or} \\
    &\Maps
      \List{f\big(\Array{1}{2}{2}\big),f\big(\Array{1}{2}{2}\big),\dots}
      \ \text{or} \\
    &\Maps
      \List{f\big(\Array{2}{2}{0}\big),f\big(\Array{2}{2}{0}\big),\dots}
  \end{align*}

\item[lapply] Applies a function over a list or vector.
  \begin{align*}
    \Array{3}{1}{0} + f
    &\Maps
      \List{f\big(\Array{1}{1}{0}\big),f\big(\Array{1}{1}{0}\big),\dots} \\
    \List{\Array{3}{1}{0},\Array{2}{2}{0},\dots} + f
    &\Maps
      \List{f\Big(\Array{3}{1}{0}\Big),f\big(\Array{2}{2}{0}\big),\dots}
  \end{align*}

\end{description}

\section{Combining}

\begin{description}

\item[rbind] Combines a sequence of vectors, arrays or data-frames by rows.
  \begin{align*}
    \Array{2}{3}{0} + \Array{1}{3}{0} + \cdots &\Maps \Array{3}{3}{0} \\
    \Dataframe{2}{v,v,v} + \Dataframe{2}{v,v,v} + \cdots
                                               &\Maps \Dataframe{3}{v,v,v}
  \end{align*}

\item[cbind] Combines a sequence of vectors, arrays or data-frames by
  columns.
  \begin{align*}
    \Array{3}{2}{0} + \Array{3}{1}{0} + \cdots &\Maps \Array{3}{3}{0} \\
    \Dataframe{3}{v,f} + \Dataframe{3}{f,v} + \cdots
                                               &\Maps \Dataframe{3}{v,f,f,v}
  \end{align*}

\item[simplify2array] Combines a list of arrays into a single array by
  adding more dimensions.
  \[ \List{\Array{3}{3}{0},\Array{3}{3}{0},\dots} \Maps \Array{3}{3}{2} \]

\item[unlist] Coerces a list of arrays to a vector.
  \[ \List{\Array{3}{2}{0},\Array{2}{2}{0},\dots} \Maps \Array{3}{1}{0} \]

\item[as.data.frame] Coerces a list of vectors to a data-frame.
  \[
  \List{\Array{3}{1}{0},\Factor{3},\Array{3}{1}{0},\dots} \Maps
  \Dataframe{3}{v,f,v}
  \]

\item[stack] Coerces a sequence of vectors into a single-column data-frame
  along with a factor.
  \[ \List{\Array{3}{1}{0},\Array{2}{1}{0},\dots} \Maps \Dataframe{3}{v,f} \]

\item[unsplit] Combines a sequence of vectors or data-frames into a single
  vector or data-frame by interleaving rows according to a factor.
  \begin{align*}
    \List{\Array{3}{1}{0},\Array{2}{1}{0},\dots} + \Factor{3}
    &\Maps \Array{3}{1}{0} \\
    \List{\Dataframe{3}{v,f,v},\Dataframe{2}{v,f,v},\dots} + \Factor{3}
    &\Maps \Dataframe{3}{v,f,v}
  \end{align*}

\item[merge] Merges two data-frames by common columns.
  \[ \Dataframe{3}{f,v,v} + \Dataframe{3}{f,v} \Maps
  \Dataframe{3}{f,v,v,v} \]
\end{description}

\vfill
\vrule width3cm height.5pt

\scriptsize
Version: 1.0 \\
Copyright \copyright\ 2016 Ernest Adrogu\'e Calveras. \\
Permission is granted to copy, distribute and/or modify this document under
the terms of the GNU Free Documentation License, Version 1.3 or any later
version published by the Free Software Foundation.


\end{multicols*}

\end{document}
